\documentclass[12pt,oneside,a4paper]{article}%{amsart}
\usepackage[margin=1in]{geometry} % see geometry.pdf on how to lay out the page. There's lots.
\usepackage{graphicx}
\usepackage{hyperref}
\usepackage{xcolor}
\usepackage{enumitem}
\usepackage{caption}
\setlength\parindent{0pt}
\usepackage[round]{natbib}
\bibliographystyle{abbrvnat}
\renewcommand{\bibsection}{}

\title{\large Response to Reviewers}
\author{\normalsize Adam R. Herrington and CoAuthors}
\date{} % delete this line to display the current date

\begin{document}

\maketitle

\section*{General Remarks}

{\color{red}{We would like to thank both reviewers for taking the time to provide thoughtful feedback on our manuscript.}} \newline

\section*{Reviewer 1}

\subsection*{General comments}

Given that the different grid and dycore choices come with considerable different computational costs, I think it would be useful to include a section addressing that specific question. Some information touching on the costs appears at different places in the manuscript. It would be good to bring that all together somewhere and make a clear comparison, e.g. including a comparative table. \newline

{\color{blue}{We have added a subsection on computational costs to address the reviewers concern, in the methods section.}} \newline

The way the term surface mass balance (SMB) is defined and used in the manuscript is confusing. The different components to the SMB are not consistently applied and there are two different definitions of the SMB in use. In part this confusion is related to the fact that the model itself does not calculate all the SMB components and assumptions have to be made about the separation between snow and ice, especially in the currently discussed uncoupled case. I would suggest to start with a physical definition of SMB with all its components in an equation and then explain based on that what the model is actually doing and what constraints have to be applied. In this context it may help to describe what the model would do in a coupled setup with active CISM and then apply further constraints for the diagnostic case currently described. \newline

{\color{blue}{We thanks the reviewer for bringing this to our attention, and agree that our constant referencing of what the SMB refers to has led to some inconsistent definitions and confusion in the text. We have (1) removed attempts (other than very generally) to define the SMB (and the ablation zone) in parenthesis after introducing these terms in the introduction, (2) clarified the description of the SMB that is computed in CESM (the section ``Simulated surface mass balance"), and (3) made clear in the ``SMB analysis" section that the SMB equation we are computing in this paper is different from that computed by CESM.}} \newline

In the discussion about the different dycores, I was wondering if there could be a notion that the latitude-longitude approaches are more mature because they have a longer history. Maybe further improvements on technical aspects like filtering and optimal solutions may be expected for the FE approaches. \newline

{\color{blue}{This is a good point. A case in point would be the improvements to the flow over topography going from CESM2.0 to CESM2.2. We are still developing CAM-SE. We have elaborated on this point in the methods section (where we point to the CESM2.2 enhancements in the Appendix) and the conclusions section, the second to last paragraph.}} \newline

The relationship between atmospheric grid and land grid seems to be important. Would it be possible to say that higher resolution of the atmospheric grid improves the precipitation, while increased resolution of the land model improves the melt calculation/ablation components? \newline

{\color{blue}{Based on my understanding of how the atmosphere and land models are coupled, I don't think running the land model at a higher resolution (and holding the atmosphere resolution fixed) will alleviate the melt biases that tends to occur with partially ice covered grid cells. This is because it is the averaging of the land state over the atmospheric grid cell that is responsible for homogenizing the state. Even if the land cells are a lot smaller, and even binary in ice cover, If the fractional ice cover within an \textit{atmospheric grid cell} is still small, then we think that the atmospheric state that is then passed back to the land model will still be unduly influenced by the non-ice portions of the atmosphere grid cell.}}


\subsection*{Specific comments}

l12 I am confused about the use of the terms "uniform resolution" and "equivalent resolution". Since the lat-lon grids are also uniform (they have 1 or 2 degree resolution everywhere) I find this is not a good description to distinguish them here. Maybe this requires a few more words to make it clear to the reader what is meant here. The confusion is not fully removed in the main text either. See later comments.\newline

{\color{blue}{We take your point and agree that this is confusing. To clarify this, we've replaced ``uniform resolution" with ``quasi-uniform unstructured grids" and clarify that ``uniform" refers to isotropic grids where every grid cell has the same length in meters, as opposed to an angular distance in spherical coordinates. Whereas we had previously used ``uniform grids" to refer to both lat-lon and quasi-uniform unstructured grids, we replace ``uniform" with ``lat-lon and quasi-uniform grids" throughout the manuscript, to be more explicit about the differences between the SE and lat-lon grids.}} \newline

l29 It would be good to add one sentence about the cost here. E.g. "... recovering resolution lost in the transition to quasi-uniform grids, albeit at increased computational cost." \newline

{\color{blue}{Done.}} \newline

l37 "latitude-longitude grids" \newline

{\color{blue}{Done.}} \newline

l75 "Dycores built on lat-lon grids have some advantages over dycores built on unstructured grids" \newline

{\color{blue}{Done.}} \newline

l92 "have large" --> "has a large" \newline

{\color{blue}{Done.}} \newline

l94 The logic of "Conversely" is not clear, since I think it is referring to the first sentence in the paragraph not the preceding one. Reformulate? \newline

{\color{blue}{Sentence beginning with conversely has been reformulated to improve clarity.}} \newline

l107 Move "(SMB; the integrated sum of precipitation and runoff)" before "of the Greenland ice sheet". \newline

{\color{blue}{Done.}} \newline

l104-110 With the benefit of a high resolution land model in mind, is it possible to run the land model at a different/higher resolution than the atmosphere model? If so, this could be mentioned here. \newline

{\color{blue}{As discussed above, I'm not optimistic about the utility of a higher-res land model, and would rather not go on a tangent about custom configurations.}} \newline

l117 "(marginal regions where seasonal melting exceeds the annual mass input from precipitation)" This sentence is not correct. What defines the ablation area is a negative SMB. Problems with the statement are that melt can refreeze and precipitation can occur as snow and rain. It would maybe help to write out the whole SMB in an equation if there is confusion about these terms. \newline

{\color{blue}{Replaced with ``marginal regions where the annual SMB is negative."}} \newline

l138 "is represented with 1/4 resolution" I believe they have used two different levels of refinement (0.5 and 0.25 degree): Maybe state "is represented with up to 1/4 resolution" instead. Or give resolution in km as they do in most of the paper: 55 km and 28 km. \newline

{\color{blue}{Used reviewers ``up to" suggestion.}} \newline

l159 What I miss in this section is a short introduction to CESM. Can be brief, but should bring the reader up to speed what kind of model that is \newline

{\color{blue}{Included a short introduction to CESM. Also moved language from the experimental design section in here, as well as include a statement that the ice sheet model is not dynamically active. Our hope is that by including this at the very top of the methods section, this partly addresses the reviewers concern that it is not clear we are not running with an active ice sheet.}} \newline

l174 Is it correct to say the dycore is a model? \newline

{\color{blue}{Reworded.}} \newline

l203 "the SE numerics have been enhanced". Enhanced compared to what. An earlier model version, I suspect. Mention that. \newline

{\color{blue}{Added: relative to the CESM2.0 release.}} \newline

l214 "... has been modified to scale the smoothing radius by the local element size, resulting in rougher topography in the refinement zone." This seems in contrast to the very idea of refinement. Maybe this should be mentioned here? \newline

{\color{blue}{I don't follow. How is rougher topography in the refinement zone inconsistent with the very idea of refinement? From our perspective it is a requirement of refinement -- finer dynamics permits rougher topography.}} \newline

l215 "For SE grids". The grids are not SE! It is clear what you mean, but good to use proper terminology. Maybe "When using the SE dycore with quasi-uniform element sizes ..." \newline

{\color{blue}{Used the reviewers preferred wording.}} \newline

l215-219 I seem to understand from this paragraph that CSLAM is an improvement, but that it cannot be applied for VR grids. If so, please make that clear here. \newline

{\color{blue}{Added. Yes, this is unfortunately true. We do have plans to address this in VR within the next 2 years.}} \newline

l220-224 I am wondering if this separate grid could in principle be extended to match an underlying SE dynamics VR grid by further subdivision into more control volumes. \newline

{\color{blue}{The underlying issue is that the SE dynamics don't have control volumes. They have GLL nodes, which are point samples of a spectrally truncated state. In order to couple SE to other model components, we had to invent control volumes because the CESM infrastructure uses finite-volume mapping algorithms to couple to other model components. The resulting control volume grid is an anisotropic mess (see Figure 4 in my MWR 2019 paper), which was constructed by constructing control volumes consistent with each GLL node's quadrature weight. It is much preferable to split the elements into quadrilateral control volumes and use the continuous basis representation to project the SE solution onto the quadrilateral control volumes (like we do in pg3 and pg2.}} \newline

l224 add "at" after "evaluated" \newline

{\color{blue}{Done.}} \newline

l249 I don't think "RACMO" has been defined yet. Please do. \newline

{\color{blue}{Done.}} \newline

l251 The discussion about computational costs is an important one in this comparison and should get at least an own paragraph, if not an own sub-section. \newline

{\color{blue}{See introductory remarks.}} \newline

l288 It is not the grids that calculate the SMB. Maybe "CESM simulates the GrIS SMB as the sum of mass accumulation and ablation" \newline

{\color{blue}{We have replaced with the suggested language. In order to reconcile your following comments regarding this paragraph, what follows this top sentence is a description of how the SMB is computed in CESM. This requires some changes to the language, such as excluding rain from the runoff terms, and making clear only ice accumulation is contained in the accumulation terms. Having addressed some unclear statements about the definition of the SMB in the introduction, we don't feel that we need to add another equation for CESM's SMB, or a common glaciological definition plus a simplified version used by CESM. These equations can be found in the references in the sentence "For a more detailed description of the SMB is computed in CESM ...", in particular equations 4 and 5 in van Kampenhout et al. 2020. }} \newline

l289 Strictly speaking only snow accumulation is a definite positive mass source, while rain can either runoff or refreeze. Again, an equation that gives all the terms in the SMB may be useful here. \newline

{\color{blue}{See above comment.}} \newline

l290 Suggest to replace "with runoff being a combination of" by "with runoff having sources of a combination of". Again, I think runoff is not a straight sum of these, because of refreezing. \newline

{\color{blue}{This has been modified in the reworked section.}} \newline

l300 Mention here that this can change if the Greenland ice sheet is fully coupled. I generally miss already earlier a statement that you are using CESM without interactive Greenland ice sheet. \newline

{\color{blue}{This is a good catch and the language has been inserted about how this changes with an active ice sheet.}} \newline

l305 These descriptions apply to the uncoupled case, which should be clarified. It may also be interesting to explain what happens in the coupled case, since CESM has that as an option. I am a bit surprised that this is not the primary option you are describing here. \newline

{\color{blue}{After addressing your last comment, we believe the paragraph is valid for both coupled and un-coupled cases.}} \newline

l308 I think it would be interesting to show that downscaled SMB (using the elevation classes) and compare that between the different cases. That is the SMB product one would actually use either in the coupled case or when extracting the SMB from CESM for offline forcing of an ice sheet model. In addition, you could compare that result more easily between the different cases as they would all be well-defined on the output grid. \newline

{\color{blue}{We agree that would be interesting. The reason we didn't employ this approach -- which to be clear, we interpret as evaluating the SMB field downscaled to the 4km CISM grid -- is that only the SMB is downscaled. It is the resolution sensitivity of the individual components of the SMB that is the main focus of this study. It's representation on the CLM grid is still quite advanced, owing to the EC scheme. I suppose we could have tried to hack the code to downscale the SMB components to CISM, but we didn't go down that road (and no telling the number of issues that would've cropped up going down that road). That said, our approach of mapping to the common ice mask is quite complicated as well.}} \newline

l313 "Changes in ice depth, but not snow depth, count toward the SMB" seems to be in contrast with "Any snow above the 10 m cap contributes towards ice accumulation in the SMB" in the paragraph before. Please clarify. \newline

{\color{blue}{I'm not sure I see the contradiction the reviewer is referring to, but this a clearly a redundancy that we've eliminated by consolidating the paragraphs.}} \newline

l322 The datasets you use are not observational but model results. I think this should be made clear already in the heading. Maybe "Target" or "Validation" datasets. \newline

{\color{blue}{Done.}} \newline

l323 Again, the datasets in use are not observational. \newline

{\color{blue}{Done.}} \newline

l323-338 You do not seem to describe anywhere the other datasets in Table 2: ERA, CERES, Calipso. These should all be described here as well. \newline

{\color{blue}{Done.}} \newline

l331 add "e.g." after before "maintained". This is a very specific collection of the data. \newline

{\color{blue}{I'm not sure I follow, but I did clarify the second category to be less specific (remote sensing datasets instead of radar accumulation datasets).}} \newline

l343 "reference dataset" Remind us what this is and where it is sourced from. \newline

{\color{blue}{Clarified the reference dataset and added citation.}} \newline

l348 "can lead to large differences in integrated SMB" Could explain briefly why that is, because the mass loss happens in a narrow band along the margins. \newline

{\color{blue}{The Hansen paper indicates that differences in the ice masks occur in its representation of the ice margins, where lots of melting conceivable occurs. So an erroneously small ice mask may overestimate the SMB due to omitting areas where melting occurs. }} \newline

l363 Equ 1. Could be read as "evaporation divided by sublimation". Why not "SMB = accumulation + runoff + evaporation + sublimation"? \newline

{\color{blue}{Done.}} \newline

l369 This is confusing, because we are in section 2.4. If you are operating with two different SMB terms, you should have two equations to clearly distinguish them. \newline

{\color{blue}{We have reworded and moved this last sentence, indicating that the total water SMB is different than the SMB internally computed by CLM, as a topic sentence to its own paragraph immediately following. We've added clarification on the role of the internally generated CLM SMB, that we don't use this definition in this study.}} \newline

l372 "grid-cell mean quantities" Clarify that these are atmospheric/land-model grid cells. \newline

{\color{blue}{Apologies for not understanding the suggestion here. What is it that needs clarification? A topic sentence in a paragraph in the methods section begins with ``CLM runs on the same grid as the atmosphere..." Is that what you are asking about?}} \newline

l388 insert "further" before "diversify". \newline

{\color{blue}{Done.}} \newline

l439 Since you start the section in 404 with "Before delving into the simulated Arctic climate conditions ...", could mark that we are now moving from global to Arctic discussion. \newline

{\color{blue}{Good suggestion. Done}} \newline

l461 "CERES product" needs to be described in 2.5. \newline

{\color{blue}{Done.}} \newline

l478 Add "radiation" after "shortwave". \newline

{\color{blue}{Done.}} \newline

l525 "underline"? \newline

{\color{blue}{Yes. Fixed.}} \newline

l528 Did you test how much biased the results could be due to the relatively short period of 20 years in view of the large inter-annual variability? If you don't have more years, you could remove any of the 20 years and see how much the numbers change. \newline

{\color{blue}{We've done performed a Monte Carlo analysis similar to what is suggested and have found that the errors due to inter-annual variability (in Figure 10) are much smaller than our current error bars, and have therefore omitted these estimates.}} \newline

l553 "smaller melt-rates" or smaller melt-rate biases? \newline

{\color{blue}{since both are true, replaced with ``smaller melt-rates and melt-rate biases."}} \newline

l564 Often PDD schemes have different PDD factors for snow and ice melt. Is this not not the case here? \newline

{\color{blue}{To simplify the calculation I just took an intermediate value between snow and ice (Braithwaite 1984 uses 3 mm/d/C and 7 mm/d/C for ice), so I wouldn't need to approximate how to partition at particular monthly average grid cell. Added these details in the manuscript in parenthesis.}} \newline

l557-564 Description and motivation for using the PDD model should probably be moved/added to the Methods section. \newline

{\color{blue}{Our personal opinion is that defining it on the fly, naturally arising from the analysis to explain a discrepancy between two contrasting results, is preferable instead of premeditating it earlier on in the manuscript.}} \newline

l571 Consider reordering figures, since we are jumping back from 11 to 9 here. \newline

l594 This reads like a contradiction to me "coarser models ... depress melt rates, which is the opposite of the melt bias that occurs in the coarse grid simulations". What happens in course models is the opposite to what happens in coarse models? \newline

{\color{blue}{Reworded this slightly to improve the clarity. ``This suggests that coarser models will tend to elevate the ablation zones relative to where they should be, which may be expected to cause anomalous (adiabatic) cooling and depressed melt rates, but this is opposite the melt bias that occurs in the coarse grid simulations."}} \newline

l603 "volume averaging" I would have guessed this was done by area averaging. \newline

{\color{blue}{Fair point. Changed to area averaging.}} \newline

l605 Maybe add "and more melting" after "warm bias over the ice-covered patch". \newline

{\color{blue}{Done.}} \newline

l622 Define PDFs. \newline

{\color{blue}{Done.}} \newline

l648 Add "refined over the US" after CONUS. \newline

{\color{blue}{Done.}} \newline

l735 hydrid --> hybrid \newline

{\color{blue}{Done.}} \newline

l748 Replace "shown" by "given" to avoid repeat of "showing/show". \newline

{\color{blue}{Done.}}

\subsection*{Figures}


\subsubsection*{Fig 1}
I think the grid lines could be thinner. On a printout this doesn't come out well. \newline
The color choice puts a lot of emphasis on the lakes in blue. Suggest to display the land-sea mask instead, i.e. make the lakes part of the land mask. \newline

{\color{blue}{Thinned the lines by half and changed the lake colors to tan/land.}} \newline

I would also consider using another projection that includes all of Greenland and it's surroundings. \newline

{\color{blue}{We have decided to keep the original polar projection, in order to show the polar points and Greenland.}} \newline

Caption. It is confusing to refer to all 4 grids as uniform, because they are uniform in different ways. At least distinguish them as uniform 1, 2 latitude-longitude grids (top) and quasi-uniform element size grids (bottom), or similar. \newline

{\color{blue}{Changed to new labeling discussed above.}}

\subsubsection*{Fig 10}
Better use y-axis label ICE+SNOW MELT consistent with Fig 11 and to avoid confusion ICE divided by SNOW. \newline

{\color{blue}{Done.}} \newline

\subsubsection*{Fig 11}
The sign convention seems to be changed from Fig 10 with positive melt. Make consistent. \newline

{\color{blue}{Done.}} \newline

\subsection*{References}

Many references are not formatted correctly in terms of upper case / lower case. There are also other problems like incomplete author lists and formatting problems. Please revise. Some examples below. \newline

l793 This is a regular paper citation that doesn't need a "Retrieved from" entry. Author list incomplete. \newline

{\color{blue}{Done.}} \newline

l797 Incomplete author list. \newline

l924 Refer to final paper. Note title change: NCAR Topo v1.0 NCAR global model topography generation software for unstructured grids. https://doi.org/10.5194/gmd-8-3975-2015 \newline

{\color{blue}{Done.}} \newline

l988 This is a regular paper citation that doesn't need a "Retrieved from" entry. Author list with "...". \newline

l1063 Refer to final paper. https://doi.org/10.5194/gmd-14-5023-2021 \newline

{\color{blue}{Done.}} \newline

l1071 Refer to final paper. https://doi.org/10.5194/tc-13-1547-2019 \newline

{\color{blue}{Done.}}

\section*{Reviewer 2}

\subsection*{General comments}

The manuscript is well-written and provides an important contribution to understanding simulation of ice sheet SMB in global climate model simulations. The paper provides important documentation of the impact of changes to the grid and spatial resolution in CESM on ice sheet SMB. Some points below should be addressed before the manuscript is published but I feel these are minor overall. \newline

One general comment: surface albedo has been shown to be an important factor in climate model simulations of GrIS SMB (e.g. van Angelen et al., 2012; Cullather et al., 2014; Helsen et al., 2017; Alexander et al., 2019), but this has not been discussed by the authors. Some of the melt differences could be influenced by the interaction between precipitation, surface albedo, and melt. In areas of underestimated snowfall, which tend to occur along the ice sheet margins, erroneous bare ice exposure can lead to a darker surface and enhanced melt. IThe authors should examine the possibility of differences in albedo influencing simulated melt. \newline

%Cullather, R. I., Nowicki, S. M. J., Zhao, B., & Suarez, M. J. (2014) Evaluation of the surface representation of the Greenland Ice Sheet in a General Circulation Model. J. Climate, 27 (13), 4835-4856. \newline

%Helsen, M. M., van de Wal, R. S. W., Reerink, T. S., Bitanja, R., Madsen, M. S., Yang, S., Li, Q., & Zhang, Q. (2017) On the importance of the albedo parameterization for the mass balance of the Greenland ice sheet in EC-Earth. The Cryosphere, 11, 1949-2017. \newline

%Alexander, P. M., LeGrande, A. N., Fischer, E., Tedesco, M., Fettweis, X., Kelley, M., Nowicki, S. M. J., & Schmidt, G. A. (2019) Simulated Greenland surface mass balance in the GISS ModelE2 GCM: Role of the ice sheet surface. J. Geophys. Res., 124 (3), 750- 765.

{\color{blue}{We thank the reviewer for this point. We had not thought too much about albedo and so it's for the best that this was brought up. This is an interesting hypothesis, that in the coarser grids, the negative precip biases along the ice sheet margins may lead to more frequent bare ice exposure, and through it's lesser albedo, more melt. We have explored whether there is any evidence for this in our simulations, and do find that there is a negative albedo anomaly around the southeast, southwest and northwest coasts in the coarse grids. However, this is only with respect to the RACMO simulations.}}

\subsection*{Specific comments}

1. Lines 19-24: The language is confusing here. The quasi-uniform grids should be introduced with a brief description of how they are constructed, and what “quasi- uniform” means. Then their performance in terms of surface mass balance can be discussed, followed by the impact of grid refinement on the simulation. Also perhaps it can be mentioned here how one dycore is linked to two lon-lat configurations and another is linked to the quasi-uniform unstructured grid. \newline

{\color{blue}{We agree that this is confusing and have added a sentence describing the different grids/dycores in more detail, before talking about their SMB performance.}}. \newline

2. Line 38: Again, please explain the meaning of “quasi-uniform unstructured grid”  \newline

{\color{blue}{See above.}} \newline

3. Lines 52-55: Again, it would be helpful if “globally uniform unstructured grid”, “dynamical cores”, and “variable-resolution grids” can be briefly explained. Additionally, perhaps it should be explained here that two dycores are examined, one linked to the longitude latitude grid, and the other linked to unstructured grids.  \newline

{\color{blue}{Added clarification for the meaning of quasi-uniform and variable-resolution. Decided against getting into the specifics of which dycore uses which grid, as this becomes apparent later on in the introduction}}. \newline

4. Line 94: Why is “conversely” used here? It seems like a word such as “Additionally” would be more appropriate.  \newline

{\color{blue}{Sentence has been reworded.}} \newline

5. Lines 115-116: Clarify that this is a positive precipitation bias in the interior.  \newline

{\color{blue}{Done.}} \newline

6. Lines 111-123: Here there is no mention of elevation classes. I suggest mentioning them briefly here as this is a way to increase SMB resolution (from the standpoint of runoff but not precipitation) that is used in CESM  \newline

{\color{blue}{We agree that the idea of ECs should be introduced here, and have added a sentence.}} \newline

7. Figure 1, caption: Please provide a brief description of each of the grids here for the benefit of the reader. E.g. (a) 2 latitude-longitude grid (f19), etc.  \newline

{\color{blue}{Done.}} \newline

8. Line 189: Briefly explain “cubed-sphere”.  \newline

{\color{blue}{Done.}} \newline

9. Table 1: It would be helpful to specify which grids are latitude-longitude and which are unstructured either as a column in the table or text in the caption.  \newline

{\color{blue}{Added to the text which grids are lat-lon and which are unstructured.}} \newline

10. Lines 229-230: Does ne30 indicate 30 quadrilateral elements per face and pg3 indicate 3 control volumes per element? Please clarify. \newline

{\color{blue}{The text indicates that ne30 indicates are 30x30 elements per cube-face, and pg3 refers to 3x3 control volumes per element. The global number of grid columns is then 6 faces * 30x30 elements per face * 3x3 control volumes per element.}} \newline

11. Lines 239-243: I suppose degree coordinates here correspond the latitude dimension, or degrees at the equator. Please clarify.  \newline

{\color{blue}{Done.}} \newline

12. Line 249: This is the first time RACMO is mentioned. Please explain briefly what RACMO is here.  \newline

{\color{blue}{Fixed.}} \newline

13. Line 254: Clarify that “expensive” means “computationally expensive.” \newline

{\color{blue}{This information has been absorbed into a new sub-section on computational costs. References to ``cheaper" or ``more expensive" are more obviously referring computational costs as that is the title of the section.}} \newline

14. Line 301: Though not essential it would be interesting to know the lapse rate that is assumed. \newline

{\color{blue}{6.5K/km. I'd refer to reviewer to the Sellevold et al paper referenced, for more details on the choice of lapse rate.}} \newline

15. Line 314: I’m confused as to why accumulation below 10 m does not contribute to SMB. How is this addition of mass accounted for? \newline

{\color{blue}{The internally computed SMB is a bit strange, in that only tracks ice. It does not consider the overlying 10m of snow to contribute to the SMB. In response to the other reviewer, we've clarified this in sections 2.5.}} \newline

16. Lines 318-319: What is the period for the prescribed SSTs? \newline

{\color{blue}{Perpetual 1979. Added this in parantheses.}} \newline

17. Lines 322-338: This section only includes discussion of RACMO and in-situ measurements, which are not used, while ERA5, CERES and CALIPSO are not at all discussed. These validation products must be documented here as well. \newline

{\color{blue}{Fixed.}} \newline

18. Line 322: RACMO2 is still a model, and does not fall under an “observational” dataset. I suggest changing this to read “Validation Datasets”, “Evaluation Datasets” or “Comparison Datasets”. \newline

{\color{blue}{Changed to validation dataset throughout the manuscript.}} \newline

19. Figure 3 caption: I believe the left and right figures are switched here. Also, the fact that RACMO RCM simulations are included should be mentioned. \newline

{\color{blue}{Fixed, and included references to the RACMO grids in the text and figure caption.}} \newline

20. Formula 1: The f’s in runoff seem to have been turned into function symbols. Also I suggest using “evaporation + sublimation” to avoid confusion with division. \newline

{\color{blue}{Switched to type writer font. Used reviewers suggestion to replace slash with a plus sign.}} \newline

21. Line 406: I believe this should read “expressed as temperature differences”. 

{\color{blue}{Did something similar, as temperature was missing from the sentence.}} \newline

22. Lines 408-416, 429-433: How can the effect be primarily ascribed to condensational heating if the temperature differences do not align in magnitude or spatial extent? Are there feedbacks involved? Please clarify in the text, or qualify the statements. \newline

{\color{blue}{The imperfect alignment of the clubb heating anomalies and the temperature anomalies has a lot to do with averaging the clubb heating over all time. If you look at Figure 10 in Herrington and Reed QJRMS paper, the alignment is strongest when you sample the clubb heating when/where the dycore has upward vertical velocities. Whereas there is no correspondence with the temperature anomalies for descending grid columns. So since our clubb heating is just a climatology, we lose that alignment with the temperature anomalies. But it's clear that a strong heating signal is piercing through in the climatology anyway, and so I'm strongly implying that this is consistent with QJRMS mechanism. I've added an entire paragraph discussing this misalignment.}} \newline

23. Lines 437-438: This also seems to be true in other cases. \newline

{\color{blue}{See above comment.}} \newline

24. Lines 440-441: This seems consistent with other results that increasing resolution led to a temperature increase. \newline

{\color{blue}{This response appears to be different from the other cases; with clear changes to the planetary waves that are not seen in the other simulations.}} \newline

25. Line 459: Perhaps the authors could mention the potential influence of clouds earlier as prior to this point, they are not mentioned. \newline

{\color{blue}{See above comment.}} \newline

26. Line 479: Clarify that cooler temperatures in the coarse resolution simulations are being referred to here. \newline

{\color{blue}{Decided to leave it as.}} \newline

27. Figure 8: The CALIPSO grid appears to be coarser resolution than the CERES grid. Is this a 2 grid? \newline

{\color{blue}{Yes, CALIPSO is on a 2 degree grid. Corrected Table 2. Great catch!}} \newline

28. Lines 573-575: Wouldn’t the elevation class scheme fix this problem by raising the ice-covered sub-grid portion to a higher elevation? \newline

{\color{blue}{That's a good point that we overlooked. We have added a sentence to the end of the next paragraph clarifying that differences between the grid cell mean and actual ice surface are not necessarily problematic due to the EC scheme.}} \newline

29. Lines 606-616: If the biases are larger at low elevations, it could also be coincidental that these areas also have lower ice fractions. I think the authors should also consider the possibility of other factors. \newline

{\color{blue}{We've added references to this coincidence in response to your related questions in the general remarks.}} \newline

30. Lines 675-680: Perhaps the fact that clouds and precipitation move more easily inland could reduce snowfall and increase shortwave radiation along the coast, enhancing melt. \newline

{\color{blue}{See response to general remarks.}} \newline

31. Lines 681-688: Again, I’m curious about the impact of elevation classes and whether precipitation-albedo feedbacks could influence the melt biases. I suppose the elevation classes might not entirely balance out other coarse- resolution effects. \newline

{\color{blue}{See response to general remarks.}} \newline

32. Line 719: Change “spectra-element” to “spectral-element”. \newline

{\color{blue}{Done.}} \newline

33. Line 729: g can be defined after mentioning T1 for clarity. \newline

{\color{blue}{Done.}} \newline

34. Line 731: Add “is the” before “surface geopotential”. \newline

{\color{blue}{Done.}} \newline

35. Equation A5: How is T(ref) computed? \newline

{\color{blue}{Defined in equation A1.}} \newline

36. Lines 740-748: Here some terms should be defined, including Tv, qv and p0. \newline

{\color{blue}{Added definitions for Tv and thetav. p0 is defined just below equation A3. I don't see a ``qv".}}

%\section*{\normalsize References}
%\setlength{\bibsep}{0pt}
%{\footnotesize
%\bibliography{bib}}


\end{document}